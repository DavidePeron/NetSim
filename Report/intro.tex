% !TEX root = template.tex

\section{Introduction}
\label{sec:introduction}

Vehicular communication will be a new promising technology that will be introduced with fifth generation cellular networks. In the near future, vehicles will communicate each other to share information regarding the surrounding environment or they will download information from Internet to allow services such as video streaming or navigation for autonomous cars.
In this scenario, each vehicle will send a lot of data over the network (in the order of terabytes per driving hour) and nowadays communication systems are not able to manage such a big amount of information \cite{Va2016}.

For this reason, several technologies (novel or well studied) have been proposed to replace or to support classic standards. Unfortunately each technology has pros and cons, mainly due to physical limitations, therefore the most recent researches try to make different technologies cohexist in order to exploit the efficiency of the network using the best technology in each moment \cite{Giordani16}.

This paper compares classical 4G communication (LTE), with \gls{mmWaves} communication and \gls{dsrc}, two technologies that use a different spectrum and that offer a very different service, to find their key features and weaknesses while applied to a vehicular scenario.
For \gls{mmWaves} and LTE a \gls{v2i} scenario is deployed in which a \gls{ue} communicates with its eNodeB while it is moving away from it. In the simulation, a dense urban scenario is considered.
For \gls{dsrc}, a \gls{v2i} communication does not make sense since this type of technology is used mostly for direct communication, therefore a \gls{v2v} situation has been implemented.
The results show how each technology can be used for specific applications depending on the amount of data shared and the required robustness of the communication.

In the next section the state-of-the-art of the multi-connectivity problem is briefly reported, then in \secref{sec:simulation_framework} the structure of the simulator used is described, in \secref{sec:results} the obtained results are showed and commented and finally in \secref{sec:conclusions} conclusions are drawn.